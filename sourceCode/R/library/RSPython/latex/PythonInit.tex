\Header{.PythonInit}{Initialize the Python interpreter.}
\alias{.PythonTerminate}{.PythonInit}
\keyword{Inter-system Interface}{.PythonInit}
\keyword{Python}{.PythonInit}
\begin{Description}\relax
These functions allow one to initializes the Python engine and creates an interpreter
and also to discard it and its contents.
The \code{.PythonInit} function  must be called before invoking a Python function or method.
One can start, stop and restart the Python interpreter any number of times.\end{Description}
\begin{Usage}
\begin{verbatim}
.PythonInit(path=NULL, merge=T)
.PythonTerminate()
\end{verbatim}
\end{Usage}
\begin{Details}\relax
\code{.PythonInit} passes its arguments
to \code{\Link{.PythonPath}} to determine the
appropriate value for the \code{PYTHONPATH} environment
variable. It then uses the returned value to set
this environment variable and then initializes the python
interpreter.\end{Details}
\begin{Value}
NULL or an error occurs.\end{Value}
\begin{Author}\relax
Duncan Temple Lang\end{Author}
\begin{References}\relax
\url{http://www.omegahat.org/RSPython},
\url{http://www.python.org}\end{References}
\begin{SeeAlso}\relax
\code{\Link{.PythonPath}}\end{SeeAlso}
\begin{Examples}
\begin{ExampleCode}

  .PythonInit()

\end{ExampleCode}
\end{Examples}

