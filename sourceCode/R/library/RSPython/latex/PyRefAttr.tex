\Header{[[.PythonReference}{~~function to do ... ~~}
\keyword{~kwd1}{[[.PythonReference}
\keyword{~kwd2}{[[.PythonReference}
\begin{Description}\relax
~~ A concise (1-5 lines) description of what the function does. ~~\end{Description}
\begin{Usage}
\begin{verbatim}
[[.PythonReference(x, i, ..., .convert=T)
\end{verbatim}
\end{Usage}
\begin{Arguments}
\begin{ldescription}
\item[\code{x}] the \code{PythonReference} object in which to find the field
named \code{name}.
\item[\code{i}] a character vector specifiying the name of the field to retrieve.
\item[\code{...}] ignored, but present for compatability with the generic.
\item[\code{.convert}] a logical value indicating whether to attempt to
convert the resulting Python object to R (TRUE) or leave as a reference (FALSE)
\end{ldescription}
\end{Arguments}
\begin{Value}
The value of the Python field
named \code{name} in the specified object,
converted to an R object or exported to
R as a reference to the Python object.\end{Value}
\begin{Note}\relax
One can get the names of the different
fields that are accessible in a Python reference
\code{x} using the command
\code{x[["\_\_dict\_\_", .convert=F]]\$keys()}\end{Note}
\begin{Author}\relax
Duncan Temple Lang\end{Author}
\begin{References}\relax
\url{http://www.omegahat.org/RSPython},
\url{http://www.python.org}\end{References}
\begin{SeeAlso}\relax
\code{\Link{\$.PythonReference}}\end{SeeAlso}
\begin{Examples}
\begin{ExampleCode}

  p <- .PythonNew("SDefsParser", "foo.defs", .module="Sgenerate")
  p[["functions"]]

  p[["__dict__", .convert=F]]$keys()

\end{ExampleCode}
\end{Examples}

