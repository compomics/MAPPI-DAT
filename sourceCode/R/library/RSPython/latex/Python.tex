\Header{.Python}{Invoke a Python function or instance method}
\alias{.PythonMethod}{.Python}
\keyword{Inter-system Interface}{.Python}
\keyword{Python}{.Python}
\begin{Description}\relax
These are the functions that invoke a Python function
or a method of a Python object.
They transfer the S arguments to their corresponding
Python values and invoke the identified function object.
The resulting Python value is transferred back as an 
S object using the default converters.\end{Description}
\begin{Usage}
\begin{verbatim}
.Python(func, ..., .module=NULL, .convert=T, .sig=NULL)
.PythonMethod(ref, method, ..., .convert=T, .sig=NULL)
\end{verbatim}
\end{Usage}
\begin{Arguments}
\begin{ldescription}
\item[\code{func}] usually, a string specifying the name of the Python function to invoke.
This can also be a \code{PythonReference} object which refers to a previously
``computed'' function object.
\item[\code{ref}] The Python object whose instance method is to be invoked. This is the 
\code{self} object in Python terms.
\item[\code{method}] the name of the instance method to be invoked.
\item[\code{...}] the arguments to the Python function or method, given as S objects
and converted to Python using the default converters.
\item[\code{.module}] the name of the module in which to find to find and resolve the
Python function
\item[\code{.convert}] a logical value indicating whether an attempt to convert 
the result from the Python function or method to a regular R value
should be made, or whether the object should be left on the Python side of
the interface and a reference to it returned. Avoiding conversion can be useful
when the value will be used in subsequent calls Python functions or methods.
\item[\code{.sig}] 
\end{ldescription}
\end{Arguments}
\begin{Details}\relax
These functions use the standard conversion mechanism for
translating R objects to Python values and vice versa.
These can be augmented by\end{Details}
\begin{Value}
The return value resulting from calling the Python function
or method. This is converted to an R object or 
returned as a reference  to the Python object
in the form of an object \code{PythonReference}.\end{Value}
\begin{Author}\relax
Duncan Temple Lang\end{Author}
\begin{References}\relax
\url{http://www.omegahat.org/RSPython},
\url{http://www.python.org}\end{References}
\begin{SeeAlso}\relax
\code{\Link{.PythonEval}}
\code{\Link{.PythonEvalFile}}
\code{\Link{.Perl}}
\code{\Link{.Java}}
\code{\Link{.CORBA}}\end{SeeAlso}
\begin{Examples}
\begin{ExampleCode}
 .Python("version", .module="sys")
 .Python("path", .module="sys")
\end{ExampleCode}
\end{Examples}

