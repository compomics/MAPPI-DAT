\Header{setPythonHandler}{Get or set the R to Python foreign reference handler.}
\alias{getPythonHandler}{setPythonHandler}
\keyword{Inter-system Interface}{setPythonHandler}
\keyword{Python}{setPythonHandler}
\begin{Description}\relax
The R-Python reference manager/handler is responsible for caching and
resolving R objects that are exported to Python as references to R
objects. The manager accepts these R objects and stores them with a
given unique name, and returns an object that identifies it. When the
object is used in Python, the reference is passed back to R and
resolved by this handler.  This handler is implemented as a collection
of basic R functions with different functions handling the storing and
resolving of references, calling functions on that reference.  Usually
this is a closure that shares access to the pool of objects being
exported.  These two functions allow one to get the currently
registered handler and also to set the handler.\end{Description}
\begin{Usage}
\begin{verbatim}
setPythonHandler(handler)
getPythonHandler(handler)
\end{verbatim}
\end{Usage}
\begin{Arguments}
\begin{ldescription}
\item[\code{handler}] the handler to be registered as the active one.
\end{ldescription}
\end{Arguments}
\begin{Details}\relax
Currently, the form of the handler is quite restricted.
It should be a list with the same length and functions 
in the same order as those in the default handler
returned by \code{\Link{referenceManager}}\end{Details}
\begin{Value}
Both functions return the reference handler that was registered
before the function was called. In the case of the
\code{setPythonHandler}, this is the value that is being replaced.\end{Value}
\begin{Author}\relax
Duncan Temple Lang\end{Author}
\begin{References}\relax
\url{http://www.omegahat.org/RSPython},
\url{http://www.python.org}\end{References}
\begin{SeeAlso}\relax
\code{\Link{referenceManager}}\end{SeeAlso}
\begin{Examples}
\begin{ExampleCode}

\end{ExampleCode}
\end{Examples}

