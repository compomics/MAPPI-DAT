\Header{PythonReference Element Access}{Syntactic convenience for calling a Python references's method}
\alias{\$.PythonReference}{PythonReference Element Access}
\keyword{Inter-system Interface}{PythonReference Element Access}
\keyword{Python}{PythonReference Element Access}
\begin{Description}\relax
This allows one to call a method of a Python object
from R in a simple form such as \code{pyObj\$method(arg1, arg2, ...)}.
This provides an overloading of the \code{\$} operator
for R objects of class \code{PythonReference} to return
a simplified call to \code{.PythonMethod}\end{Description}
\begin{Usage}
\begin{verbatim}
$.PythonReference(obj, name)
\end{verbatim}
\end{Usage}
\begin{Arguments}
\begin{ldescription}
\item[\code{obj}] the \code{PythonReference} object whose method is to be called.
\item[\code{name}] the name of the method in the Python object which is to be called.
\end{ldescription}
\end{Arguments}
\begin{Details}\relax
When calling a Python object's method from R, one specifies the
reference to the Python object and the name of the method along with
the arguments to that method in a call to \code{\Link{.PythonMethod}}. 
This function/method returns  function object containing sufficient
information to call the specified method on the given reference.
These two pieces of information are stored in an R closure instance
and this can then be called as a regular function whose
body calls \code{\Link{.PythonMethod}}.\end{Details}
\begin{Value}
A function closure instance whose body consists
of a call to \code{.PythonMethod} for the 
specified \code{PythonReference} object
and method name.\end{Value}
\begin{Author}\relax
Duncan Temple Lang\end{Author}
\begin{References}\relax
\url{http://www.omegahat.org/RSPython},
\url{http://www.python.org}\end{References}
\begin{SeeAlso}\relax
\code{\Link{.PythonMethod}}
\code{\Link{.Python}}
\code{\Link{.PythonNew}}
\code{\Link{.PythonEval}}
\code{\Link{.PythonEvalFile}}\end{SeeAlso}
\begin{Examples}
\begin{ExampleCode}
 x <- .PythonNew("") 
 x$foo()




\end{ExampleCode}
\end{Examples}

