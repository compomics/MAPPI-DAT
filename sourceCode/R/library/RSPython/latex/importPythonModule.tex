\Header{importPythonModule}{Loads a Python Module}
\keyword{Inter-system Interface}{importPythonModule}
\keyword{Python}{importPythonModule}
\begin{Description}\relax
This loads the Python module into the Python interpreter,
making its functions and classes available to the Python, and hence R,
session. This is equivalent to the Python command
\code{from <name> import *}.\end{Description}
\begin{Usage}
\begin{verbatim}
importPythonModule(name, all=T)
\end{verbatim}
\end{Usage}
\begin{Arguments}
\begin{ldescription}
\item[\code{name}] 
\item[\code{all}] 
\end{ldescription}
\end{Arguments}
\begin{Value}
\end{Value}
\begin{Author}\relax
Duncan Temple Lang\end{Author}
\begin{References}\relax
\url{http://www.omegahat.org/RSPython},
\url{http://www.python.org}\end{References}
\begin{SeeAlso}\relax
\code{\Link{pythonModuleTypes}}\end{SeeAlso}
\begin{Examples}
\begin{ExampleCode}
  importPythonModule("sys")
  .PythonEval("version")

  importPythonModule("sys", all=F)
  .PythonEval("sys.version")


  importPythonModule("urllib", all=F)
  pythonModuleTypes("urllib")

\end{ExampleCode}
\end{Examples}

