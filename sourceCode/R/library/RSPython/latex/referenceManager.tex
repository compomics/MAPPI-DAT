\Header{referenceManager}{Create instance of the standard Reference Manager}
\keyword{Inter-system Interface}{referenceManager}
\keyword{Python}{referenceManager}
\begin{Description}\relax
\end{Description}
\begin{Usage}
\begin{verbatim}
referenceManager()
\end{verbatim}
\end{Usage}
\begin{Details}\relax
If no reference manager is available when it is first needed
by a call from Python to R, this function is called to create
a \code{referenceManager} instance and is then registered
at the C level.\end{Details}
\begin{Value}
A closure that stores R objects and returns
references to them. The closure provides a variety of different
functions to operate on these R objects:
\begin{ldescription}
\item[\code{handler}] 
\item[\code{createReference}] 
\item[\code{addReference}] 
\item[\code{remove}] 
\item[\code{getReference}] 
\item[\code{references}] 
\item[\code{total}] 
\end{ldescription}
\end{Value}
\begin{Author}\relax
Duncan Temple Lang\end{Author}
\begin{References}\relax
\url{http://www.omegahat.org/RSPython},
\url{http://www.python.org}\end{References}
\begin{SeeAlso}\relax
\code{\Link{setPythonHandler}}
\code{\Link{getPythonHandler}}\end{SeeAlso}
\begin{Examples}
\begin{ExampleCode}

\end{ExampleCode}
\end{Examples}

