\Header{.PythonEvalFile}{Evaluate a Python script from a string or file.}
\alias{.PythonEval}{.PythonEvalFile}
\keyword{Inter-system Interface}{.PythonEvalFile}
\keyword{Python}{.PythonEvalFile}
\begin{Description}\relax
These two functions allow one to evaluate Python scripts
which are given in the Python syntax. These can be specified
as a string or as a file containing the Python code.
This approach to evaluating Python commands is less
flexible than those offered by \code{\Link{.Python}}
and related functions.\end{Description}
\begin{Usage}
\begin{verbatim}
.PythonEvalFile(cmd, ..., .module=NULL, .convert=T)
.PythonEval(cmd, ..., .module=NULL, .convert=T)
\end{verbatim}
\end{Usage}
\begin{Arguments}
\begin{ldescription}
\item[\code{cmd}] the Python expression to parse and evaluate.
\item[\code{...}] Any number of name-value pairs that are converted to a Python
dictionary or context which will be used when evaluating the expression. 
\item[\code{.module}] the name of the module whose dictionary is used to provide the name
space for resolving symbols in the Python expressions.
\item[\code{.convert}] a logical value indicating whether the result of evaluating the Python
object should be converted to a regular R object, if possible, or explicitly left
within Python and returned as a \code{PythonReference} object.
\end{ldescription}
\end{Arguments}
\begin{Details}\relax
\end{Details}
\begin{Value}
The value returned from evaluating the Python expression(s),
converted to R using the standard conversion mechanism, including
any user-level converters currently registered.\end{Value}
\begin{Author}\relax
Duncan Temple Lang\end{Author}
\begin{References}\relax
\url{http://www.omegahat.org/RSPython},
\url{http://www.python.org}\end{References}
\begin{SeeAlso}\relax
\code{\Link{.Python}}
\code{\Link{.PythonMethod}}
\code{\Link{.PythonNew}}\end{SeeAlso}
\begin{Examples}
\begin{ExampleCode}
 .PythonEval("from sys import *")
 .PythonEval("argv")
\end{ExampleCode}
\end{Examples}

