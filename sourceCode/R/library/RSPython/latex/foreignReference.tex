\Header{foreignReference}{Creates a reference for an S object}
\keyword{Inter-system Interface}{foreignReference}
\keyword{Python}{foreignReference}
\begin{Description}\relax
This function creates a reference from an S object and optionally
registers it with the foreign reference manager.  This reference can
then be used to identify the S object  in a foreign
system, passing it as either an argument in a function call
to that foreign system or as the result of a call to an
S function from that system.
The reference can be given additional information
such as a name by which to identify it, one or more
target classes in the foreign system that it should 
masquerade as.\end{Description}
\begin{Usage}
\begin{verbatim}
foreignReference(obj, id=NULL, className=NULL, targetClasses=NULL, register=T)
\end{verbatim}
\end{Usage}
\begin{Arguments}
\begin{ldescription}
\item[\code{obj}] the S object which the new reference is to identify
\item[\code{id}] an optional name to use to identify this reference and value.
This should be unique and if omitted will be generated by the reference manager
so that it is unique. It can be specified here to provide a convenient way to 
refer to an object and also to simplify garbage collection.
\item[\code{className}] ?
\item[\code{targetClasses}] the names of any classes in the foreign system 
that this reference should ``implement'' when it is instantiated in this system.
This is a way of controlling the conversion of the reference to an object
in the foreign system.
\item[\code{register}] a logical value indicating whether the reference
and object should be registered with the current foreign reference manager. 
\end{ldescription}
\end{Arguments}
\begin{Value}
An object of class 
\code{NamedRReference} or
\code{AnonymousRReference} depending on whether
\code{id} is specified or not.
The object contains the following fields
\begin{ldescription}
\item[\code{id}] the name for the reference
\item[\code{value}] the object or value (\code{obj} in the call) associated with this reference
\item[\code{className}] 
\item[\code{targetClasses}] the value of \code{targetClasses} in the call
\item[\code{type}] the type of the object being referenced, computed by \code{\Link{typeof}}
\item[\code{classes}] the class(es) of the object being referenced, computed by \code{\Link{class}}
\end{ldescription}
\end{Value}
\begin{Author}\relax
Duncan Temple Lang\end{Author}
\begin{References}\relax
\url{http://www.omegahat.org/RSPython},
\url{http://www.python.org}\end{References}
\begin{SeeAlso}\relax
\code{\Link{referenceManager}}
\code{\Link{.Python}}
\code{\Link{.PythonNew}}
\code{\Link{.PythonMethod}}\end{SeeAlso}
\begin{Examples}
\begin{ExampleCode}
\end{ExampleCode}
\end{Examples}

