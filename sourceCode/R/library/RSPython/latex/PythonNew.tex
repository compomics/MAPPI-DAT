\Header{.PythonNew}{Create an instance of a Python class}
\keyword{Inter-system Interface}{.PythonNew}
\keyword{Python}{.PythonNew}
\begin{Description}\relax
This function creates an instance of a Python class.
Generally, this returns a reference to that object
so that it can be used in subsequent Python calls 
from R.\end{Description}
\begin{Usage}
\begin{verbatim}
.PythonNew(className, ..., .module=NULL, .convert=F)
\end{verbatim}
\end{Usage}
\begin{Arguments}
\begin{ldescription}
\item[\code{className}] the name of the class of which an  new instance is to be created
\item[\code{...}] any arguments to the \code{\_\_init\_\_} method of the class, including
named arguments
\item[\code{.module}] the ``name'' of the module in which the class is defined. This is
usually a single string, but for nested modules is a character vector whose elements
are in the top-level to sub-module order; e.g. if a class C is defined in the module B
which is in A (i.e. A.B.C in Python terms), the \code{.module} is given as
\code{c("A","B")}
\item[\code{.convert}] a logical value indicating whether to attempt to convert the newly created object
or not. This is usually only \code{FALSE} when creating the object has a useful side-effect.
\end{ldescription}
\end{Arguments}
\begin{Value}
If \code{.convert} is \code{FALSE}, a reference to the 
newly created object is returned. This is an object of
class \code{}. Otherwise, an attempt to convert the newly created
object using the built-in and the current user-specified converters
is performed. The result depends on what these return. If no converter
is found, a reference to the newly created object is  returned, just
as if \code{.convert} was specified as \code{FALSE}.\end{Value}
\begin{Author}\relax
Duncan Temple Lang\end{Author}
\begin{References}\relax
\url{http://www.omegahat.org/RSPython},
\url{http://www.python.org}\end{References}
\begin{SeeAlso}\relax
\code{\Link{.PythonMethod}}
\code{\Link{.Python}}\end{SeeAlso}
\begin{Examples}
\begin{ExampleCode}
u <- .PythonNew("urlopen", "http://www.omegahat.org/index.html", .module="urllib")
.PythonMethod(u, "geturl")  
txt <- u$read()
u$geturl()
u$close()
\end{ExampleCode}
\end{Examples}

