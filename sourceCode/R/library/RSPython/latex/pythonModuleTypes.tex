\Header{pythonModuleTypes}{Python Module Elements and Types}
\keyword{Inter-system Interface}{pythonModuleTypes}
\keyword{Python}{pythonModuleTypes}
\begin{Description}\relax
This function provides reflective information about the
contents of a module by returning a list of all the
entries in that module and the type of object it (currently) is.
This allows one to differentiate the functions or methods
and the fields of a module  and to understand what the fields
contain.
Essentially it describes the elements of the module's dictionary
(\code{\_\_dict\_\_})\end{Description}
\begin{Usage}
\begin{verbatim}
pythonModuleTypes(name)
\end{verbatim}
\end{Usage}
\begin{Arguments}
\begin{ldescription}
\item[\code{name}] the name of the module in question. This is given as a single string or
a character vector containg the hierarchy of module names in a top-down or top-first form.
\end{ldescription}
\end{Arguments}
\begin{Details}\relax
\end{Details}
\begin{Value}
A named character/string vector whose name-value pairs
give the name of each  entry in the module and its type.
The types are\end{Value}
\begin{Author}\relax
Duncan Temple Lang\end{Author}
\begin{References}\relax
\url{http://www.omegahat.org/RSPython},
\url{http://www.python.org}\end{References}
\begin{SeeAlso}\relax
\end{SeeAlso}
\begin{Examples}
\begin{ExampleCode}
 pythonModuleTypes("sys")
 pythonModuleTypes("RS")
\end{ExampleCode}
\end{Examples}

