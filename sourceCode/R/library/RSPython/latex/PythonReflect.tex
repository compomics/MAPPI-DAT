\Header{Python Reflectance}{Information about Python classes and methods}
\alias{getSuperClasses}{Python Reflectance}
\alias{pythonMethods}{Python Reflectance}
\alias{pythonModuleNames}{Python Reflectance}
\keyword{Inter-system Interface}{Python Reflectance}
\keyword{Python}{Python Reflectance}
\begin{Description}\relax
These functions return information about a python class
and its methods. From this we\end{Description}
\begin{Usage}
\begin{verbatim}
pythonMethods(obj, all=F)
getSuperClasses(obj, all=F)
pythonModuleNames(obj, all=F)
\end{verbatim}
\end{Usage}
\begin{Arguments}
\begin{ldescription}
\item[\code{obj}] a character vector giving the name of a Python class,
given in the form \code{c("className", "moduleName")}
\item[\code{all}] controls how inherited methods and attributes are
processed. If \code{TRUE} is specified, the inherited methods are
also reported. Otherwise, these are ignored and only those methods
defined within the class are returned.
\end{ldescription}
\end{Arguments}
\begin{Details}\relax
This uses the code\end{Details}
\begin{Value}
\end{Value}
\begin{Author}\relax
Duncan Temple Lang\end{Author}
\begin{References}\relax
\url{http://www.omegahat.org/RSPython},
\url{http://www.python.org}\end{References}
\begin{SeeAlso}\relax
\end{SeeAlso}
\begin{Examples}
\begin{ExampleCode}

\end{ExampleCode}
\end{Examples}

