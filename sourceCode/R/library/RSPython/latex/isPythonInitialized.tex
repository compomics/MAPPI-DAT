\Header{isPythonInitialized}{Is Python Interpreter Initialized}
\keyword{Inter-system Interface}{isPythonInitialized}
\keyword{Python}{isPythonInitialized}
\begin{Description}\relax
This function tests whether the Python interpreter is in an initialized state
and optionally initializes it if it is not (using default settings).\end{Description}
\begin{Usage}
\begin{verbatim}
isPythonInitialized(init=F)
\end{verbatim}
\end{Usage}
\begin{Arguments}
\begin{ldescription}
\item[\code{init}] a logical value indicating whether to perform the default initialization
if the Python interpreter is not already initialized.

\end{ldescription}
\end{Arguments}
\begin{Details}\relax
This checks the internal state of the Python library and queries if the
interpreter is already initialized. If not and \code{init} is \code{TRUE},
the function \code{\Link{.PythonInit}} is called with no arguments.\end{Details}
\begin{Value}
A logical value indicating what the current status of the Python interpreter
is. If \code{init} is \code{FALSE}, this is the state before the call.
Otherwise, the interpreter should have been started, except if an error occurs.\end{Value}
\begin{Author}\relax
Duncan Temple Lang\end{Author}
\begin{References}\relax
\url{http://www.omegahat.org/RSPython},
\url{http://www.python.org}\end{References}
\begin{Examples}
\begin{ExampleCode}
 isPythonInitialized(init)

 if(isPythonInitialized())
\end{ExampleCode}
\end{Examples}

